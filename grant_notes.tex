\documentclass[12pt,tightenlines, raggedbottom, prd, notitlepage]{revtex4-1}

\usepackage{amsmath}
\usepackage{amssymb}
\usepackage{braket}
\usepackage{booktabs}
\usepackage{hyperref} 

% Forces bold math in section/subsection/etc headings 
\makeatletter
\g@addto@macro\bfseries{\boldmath}
\makeatother

\begin{document}


\title{Hyperon Masses}
\author{Grant Bradley}
\maketitle

\section*{xpt preliminaries}

QCD vacuum breaks chiral symmetry down to $U(3)_v$, the diagonal subgroup of SU(3).
8 massless goldstone bosons appear, each coupled via $F_0$ to conserved axial-vector current; 
Physics of the goldstone bosons, the pion fields, describe a low energy EFT called XPT 
  
One expands the physical vacuum state of QCD to obtain xpt via the treatment of the baryon fields as heavy static fermions.
In the chiral limit, where the quark masses are taken to zero, the momentum transfer between baryons by pion exchange, $p^\mu = m_Bv^\mu + k^\mu$,
is $\lll m_B$, thus the baryon velocity is conserved.

The mass splittings in SU(3) within a multiplet possess non-trivial chiral transformation properties, thus are treated as perturbations.
They can be treated as perturbations since their value is much less than the average baryon mass.





low energy structure of theory depends on size of the quark masses

heavy quarks dont play a role (except in heavy baryon xpt?) since their DOF frozen at low energies *why*

in QCD, spont. symm. breaking yields the goldstone bosons which produce pole in the 2 pt function 

flavor SU(3) realized as global symmetry of hadron spectrum 

8 parameter, cpt, is a simply connected lie group 

$h_{abc}$ is invariant under cyclic permutations i.e. $h_{abc} = h_{bca} = h_{cab}$

gauge principle generates interactions between matter fields through exchance of massless gauge bosons 

lagrangian containing only light flavor quarks in chiral limit, where $m_u, m_d, m_s \rightarrow 0$, starting pt. for low-energy QCD. 

global symms in the above lagrangian appear manifest in chirality matrix $\gamma_5$

a chiral (field) var is one under which parity is tranformed into neither orig. var nor its neg. 

trans: $x \rightarrow -x$ if $x$ is a vector


Replacement of the fermion field by a boson field leads to a free field theory 

using this field theory, compute correlation functions of \textbf{fermion bilinears} explicity.
in the language of renormalization groups, the model contains a line of fixed points parameterized by 
the coupling constant g. 

$$
  J^\mu_a = \frac{\partial \delta \mathcal{L}}{\partial \partial_\mu \epsilon_a}
  \partial_{\mu} J^{\mu}_a = \frac{\partial \delta \mathcal{L}}{\partial\ \epsilon_a}
$$

light quark mass dependence of hardron masses det. by XPT 
analytic terms depended on lECS of the chiral Lagrangian

lattice quark masses may be too large for SU(3) xpt to be valid, 
perturbative xpt behavior occurs only for $m_q < m_s$
WHY SU3 FLAVOR SYMMETRY EVIDENT IN BARYON PHENOMENOLOGY

$1/n_c$ expansion constrains strucutre of baryon xpt 
chiral corrections to the chiral lagrangian ?? have to respect spin-flavor structure of $1/n_c$ expansion 

mass relations(function of $m_q$)  project baryon masses onto diff spin-flavor channels 

sometimes perturbative qcd is called asymptotic freedom 

operator expansion for mass splittings of octet and decuplet uses quark operators as operator basis, 
can also use skyrme operator basis. what is diff??


can use lo xpt to relate quark masses to pion masses  

in isospin limit of SU(3), only 2 independent quark masses, 3 independent meson masses.
one can always convert from a quark mass expansion to a meson mass expansion via the gell-mann-okubo relation


\end{document}