\documentclass[12pt,tightenlines, raggedbottom, prd, notitlepage]{revtex4-1}

\usepackage{amsmath}
\usepackage{amssymb}
\usepackage{braket}
\usepackage{booktabs}
\usepackage{hyperref} 

% Forces bold math in section/subsection/etc headings 
\makeatletter
\g@addto@macro\bfseries{\boldmath}
\makeatother

\begin{document}


\title{Hyperon Masses}
\author{Grant Bradley}
\maketitle

\section*{xpt preliminaries}

QCD vacuum breaks chiral symmetry down to $U(3)_v$, the diagonal subgroup of SU(3).
8 massless goldstone bosons appear, each coupled via $F_0$ to conserved axial-vector current; 
Physics of the goldstone bosons, the pion fields, describe a low energy EFT called XPT 
  
One expands the physical vacuum state of QCD to obtain xpt via the treatment of the baryon fields as heavy static fermions.
In the chiral limit, where the quark masses are taken to zero, the momentum transfer between baryons by pion exchange, $p^\mu = m_Bv^\mu + k^\mu$,
is $\lll m_B$, thus the baryon velocity is conserved.

The mass splittings in SU(3) within a multiplet possess non-trivial chiral transformation properties, thus are treated as perturbations.
They can be treated as perturbations since their value is much less than the average baryon mass.

\subsection*{Poisson summation formula}
Poisson summation formula $\Rightarrow $ functional equation for \textbf{jacobi theta function}
$\Rightarrow$ that of the \textbf{Riemann Zeta function}

$\theta(s) = \sum_{n=-\infty}^{\infty}e^{-\pi n^2s}$ where $s\in\mathbb{R}$, however, one can 
also take $s\in\mathbb{C}$ and apply the Poisson summation formula to the Schwarz function
$f(x) = e^{-\pi sx^2}, \hat{f}(p) = \frac{1}{\sqrt{s}}e^{-\pi \frac{p^2}{s}} $
which is just the functional equatiion of the theta function.

The more general theta function, the Jacobi theta function, is a function of two complex
variables $\vartheta(z,\tau) = \sum_{n = - \infty}^{\infty} \exp(\pi i n^2 \tau + 2\pi i n z) \,. $

\subsection*{Why do we need finite volume corrections?}
Loop integrals get replaced with loop sums since LQCD calculations are carried out on a lattice
with finite volume $L^3$, thus, the calculated hadronic masses will have some dependence on L.
Due to the finite nature of the box with periodic boundary conditions, the spectrum of the
hamiltonian operator is then discrete; The corresponding energy values $M_i(L)$ within the spectrum
carry a L dependence. The afroementioned pointlike stable particles are accompanied by a cloud of 
virtual particles; This cloud, when ``squeezed'' by the box, causes the energy to deviate from the
infinite volume mass. The probability for a single quark to separate from its partner(s), since of course
quarks are confined and always coupled to a quark(s) via gluon(s), rapidly goes to zero as the size of L
grows. Thus, the leading finite size effect on the hadron masses as L increases is due to the ``squeezing''
of the virtual pion cloud around the particles. 

Meson pole, $p^2 = -m^2$ in the euclidean propagator has \textbf{unit residue} 

The correlation functions of $\phi$ can be expanded in a series of feynman diagrams with momentum
space propagators $\tilde{\Delta}(p;m) = (m^2 + p^2)^-1$. The feynman rules for finite L correlation functions
are the same as in infinite volume except that the space-like components of loop and external momenta
are restricted to discrete values $\textbf{p} = \frac{2\pi}{L}\textbf{n}, n\in \mathbb{Z^3}$

\subsection*{$\mathbb{Z^3}$ Gauge fields on a graph}

It is possible to define gauge fields on an abstract graph $\mathcal{G}$, as on regular lattices. 

In order to study the L dependence of feynman diagrams as $L\rightarrow \infty$, one works in position
space rather than momentum space. The infinite volume propagator is 
$\Delta(x;m) = \int e^{ipx}(m^2 + p^2)^{-1}\frac{d^4p}{(2\pi)^4}$

bound states give rise to poles in the analytically continued forward elastic meson scattering amplitude 










low energy structure of theory depends on size of the quark masses

heavy quarks dont play a role (except in heavy baryon xpt?) since their DOF frozen at low energies *why*

in QCD, spont. symm. breaking yields the goldstone bosons which produce pole in the 2 pt function 

flavor SU(3) realized as global symmetry of hadron spectrum 

8 parameter, cpt, is a simply connected lie group 

$h_{abc}$ is invariant under cyclic permutations i.e. $h_{abc} = h_{bca} = h_{cab}$

gauge principle generates interactions between matter fields through exchance of massless gauge bosons 

lagrangian containing only light flavor quarks in chiral limit, where $m_u, m_d, m_s \rightarrow 0$, starting pt. for low-energy QCD. 

global symms in the above lagrangian appear manifest in chirality matrix $\gamma_5$

a chiral (field) var is one under which parity is tranformed into neither orig. var nor its neg. 

trans: $x \rightarrow -x$ if $x$ is a vector


Replacement of the fermion field by a boson field leads to a free field theory 

using this field theory, compute correlation functions of \textbf{fermion bilinears} explicity.
in the language of renormalization groups, the model contains a line of fixed points parameterized by 
the coupling constant g. 

$$
  J^\mu_a = \frac{\partial \delta \mathcal{L}}{\partial \partial_\mu \epsilon_a}
  \partial_{\mu} J^{\mu}_a = \frac{\partial \delta \mathcal{L}}{\partial\ \epsilon_a}
$$

light quark mass dependence of hardron masses det. by XPT 
analytic terms depended on lECS of the chiral Lagrangian

lattice quark masses may be too large for SU(3) xpt to be valid, 
perturbative xpt behavior occurs only for $m_q < m_s$
WHY SU3 FLAVOR SYMMETRY EVIDENT IN BARYON PHENOMENOLOGY

$1/n_c$ expansion constrains strucutre of baryon xpt 
chiral corrections to the chiral lagrangian ?? have to respect spin-flavor structure of $1/n_c$ expansion 

mass relations(function of $m_q$)  project baryon masses onto diff spin-flavor channels 

sometimes perturbative qcd is called asymptotic freedom 

operator expansion for mass splittings of octet and decuplet uses quark operators as operator basis, 
can also use skyrme operator basis. what is diff??


can use lo xpt to relate quark masses to pion masses  

in isospin limit of SU(3), only 2 independent quark masses, 3 independent meson masses.
one can always convert from a quark mass expansion to a meson mass expansion via the gell-mann-okubo relation


\end{document}