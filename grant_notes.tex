\documentclass[12pt,tightenlines, raggedbottom, prd, notitlepage]{revtex4-1}

\usepackage{amsmath}
\usepackage{amssymb}
\usepackage{braket}
\usepackage{booktabs}
\usepackage{hyperref} 
\usepackage{braket}

% Forces bold math in section/subsection/etc headings 
\makeatletter
\g@addto@macro\bfseries{\boldmath}
\makeatother
\def\mc#1{{\mathcal #1}}
\def\tr{\text{tr}}
\def\ol{\overline}

\begin{document}


\title{Hyperon Masses}
\author{Grant Bradley}
\maketitle

\section*{xpt preliminaries}

XPT, the low energy approximation of QCD, carries information about the spontaneous and explicit chiral symmetry breaking in QCD. The low energy structure
of the theory depends on the size of the quark masses; Heavy quarks don't play a role since their DOFs are frozen at low energies.  
The overarching goal of XPT is to extract hadron properties, in terms of universal low-energy properties in an expansion about vanishing light
quark masses, typically $m_\pi$. Flavor SU(3) is realized as a global symmetry of the hardon spectrum. In order to parameterize only the pion mass dependence of hadronic observables, SU(2) serves as a valid starting point,
thereby sidestepping the potential convergence issues inherent in SU(3). 

It is important to remark that homomorphisms of abstract groups to grpups of linear operators, symmetry groups to spaces of
physical spaces in this context, form the basis for the theory of group representations. 

QCD vacuum breaks chiral symmetry down to $U(3)_v$, the diagonal subgroup of SU(3). Eight massless goldstone bosons appear,
each coupled via $F_0$ to conserved axial-vector current, thereby producing a pole in the 2-pt function. Eight parameter XPT is a simply connected
lie group.  Physics of the goldstone bosons, the pion fields, describe a low energy EFT called XPT; 
The gauge principle generates interactions between matter fields through exchange of massless gauge bosons.   
  
One expands the physical vacuum state of QCD to obtain xpt via the treatment of the baryon fields as heavy static fermions.
In the chiral limit, where the quark masses are taken to zero, the momentum transfer between baryons by pion exchange, $p^\mu = m_Bv^\mu + k^\mu$,
is $\lll m_B$, thus the baryon velocity is conserved.

The mass splittings in SU(3) within a multiplet possess non-trivial chiral transformation properties, thus are treated as perturbations.
They can be treated as perturbations since their value is much less than the average baryon mass.

\section*{Scattered notes}


$h_{abc}$ is invariant under cyclic permutations i.e. $h_{abc} = h_{bca} = h_{cab}$



lagrangian containing only light flavor quarks in chiral limit, where $m_u, m_d, m_s \rightarrow 0$, starting pt. for low-energy QCD. 

global symms in the above lagrangian appear manifest in chirality matrix $\gamma_5$

a chiral (field) var is one under which parity is tranformed into neither orig. var nor its neg. 

trans: $x \rightarrow -x$ if $x$ is a vector


Replacement of the fermion field by a boson field leads to a free field theory 

using this field theory, compute correlation functions of \textbf{fermion bilinears} explicity.
in the language of renormalization groups, the model contains a line of fixed points parameterized by 
the coupling constant g. 

$$
  J^\mu_a = \frac{\partial \delta \mathcal{L}}{\partial \partial_\mu \epsilon_a}
  \partial_{\mu} J^{\mu}_a = \frac{\partial \delta \mathcal{L}}{\partial\ \epsilon_a}
$$

light quark mass dependence of hardron masses det. by XPT 
analytic terms depended on lECS of the chiral Lagrangian

lattice quark masses may be too large for SU(3) xpt to be valid, 
perturbative xpt behavior occurs only for $m_q < m_s$
WHY SU3 FLAVOR SYMMETRY EVIDENT IN BARYON PHENOMENOLOGY

\subsection*{\texorpdfstring{$1/n_c$}{expansion}}
$1/n_c$ expansion constrains strucutre of baryon xpt 

chiral corrections to the chiral lagrangian ?? have to respect spin-flavor structure of $1/n_c$ expansion 

mass relations(function of $m_q$)  project baryon masses onto diff spin-flavor channels 

sometimes perturbative qcd is called asymptotic freedom 

operator expansion for mass splittings of octet and decuplet uses quark operators as operator basis, 
can also use skyrme operator basis. what is diff??


can use lo xpt to relate quark masses to pion masses  

in isospin limit of SU(3), only 2 independent quark masses, 3 independent meson masses.
one can always convert from a quark mass expansion to a meson mass expansion via the gell-mann-okubo relation

Extrapolation is insensitive to hyperon couplings

Concerned with the convergence of the EFT. 

In the $S=1$ spectrum, have virtual baryons. 
$m_\pi < m_\Delta$ in order to decay. 
When $m_\pi > m_\Delta$, spin-$1/2$ is actually stable as there is no imaginary mass. Is this the correct implication

\section*{Nucleon sigma term}

The sum of sigma terms determines the coupling strength of nucleons to WIMPs via spin-independent channel. 
Sigma terms represent the contributions of explicit chiral symmetry breaking to the nucleon mass. 

Need to employ baryon xpt to extrapolate lattice data for $M_N$ with heavy quarks to the physical
and chiral limits, while including finite volume effects. Perform SU(2) baryon xpt to compute 
extrapolations of sigma terms and mass of nucleon with the light quark mass as expansion parameter. 


A nucleon (proton or neutron) couples to the external higgs field via $g_q$ where $q$
is the quark falvor. Elementary fermions are massive, $m_f = g_f\frac{\partial m_f}{\partial g_f}$ where 
$g_f$ is the fermion-higgs coupling. 

For the nucleon: $M_Nf_{qN} = \sigma_{qN} := g_q\frac{\partial M_n}{\partial g_q} = m_q\frac{\partial M_N}{\partial m_q}$

$ \frac{\partial M_n}{\partial g_q} = \Braket{N | \bar{q}q | N}$

The overarching aim is to determine $M_\pi$ dependence on $M_N$ and compute $\sigma_{\pi N}$ 
via the feynman-hellman theorem. 
\textbf{Result:} $M_N(M\pi)$

\subsection*{Derivation of the perturbative nucleon mass}







\section*{XPT expressions for Delta and Nucleon}
In order to compute the difference between the average nucleon and average delta masses (eqn. 35), we need to compute eqns. 17 and 27
(nucleon, delta mass extrapolations, respectively) and the several orders(lo,nlo,n2lo), therein? (18-22 for nucleon and 28-30 for delta)?
https://arxiv.org/abs/hep-lat/0501018 gives xpt expressions and mass expansions about $m_\pi^2$

https://arxiv.org/abs/hep-lat/0405007v2 where lecs are defined






\subsection*{simultaneous fits for Delta}






\subsection*{Poisson summation formula}
Poisson summation formula $\Rightarrow $ functional equation for \textbf{jacobi theta function}
$\Rightarrow$ that of the \textbf{Riemann Zeta function}

$\theta(s) = \sum_{n=-\infty}^{\infty}e^{-\pi n^2s}$ where $s\in\mathbb{R}$, however, one can 
also take $s\in\mathbb{C}$ and apply the Poisson summation formula to the Schwarz function
$f(x) = e^{-\pi sx^2}, \hat{f}(p) = \frac{1}{\sqrt{s}}e^{-\pi \frac{p^2}{s}} $
which is just the functional equatiion of the theta function.

The more general theta function, the Jacobi theta function, is a function of two complex
variables $\vartheta(z,\tau) = \sum_{n = - \infty}^{\infty} \exp(\pi i n^2 \tau + 2\pi i n z) \,. $

\subsection*{Why do we need finite volume corrections?}
At finite temperature, the spectral density of the correlator differs from the zero temperature case
so that it may not be dominated by the lowest intermediate state with quantum numbers of the currents 
at large t. Loop integrals get replaced with loop sums since LQCD calculations are carried out on a lattice
with finite volume $L^3$, thus, the calculated hadronic masses will have some dependence on L.
Due to the finite nature of the box with periodic boundary conditions, the spectrum of the
hamiltonian operator is then discrete; The corresponding energy values $M_i(L)$ within the spectrum
carry a L dependence. The afroementioned pointlike stable particles are accompanied by a cloud of 
virtual particles; This cloud, when ``squeezed'' by the box, causes the energy to deviate from the
infinite volume mass. The probability for a single quark to separate from its partner(s), since of course
quarks are confined and always coupled to a quark(s) via gluon(s), rapidly goes to zero as the size of L
grows. Thus, the leading finite size effect on the hadron masses as L increases is due to the ``squeezing''
of the virtual pion cloud around the particles. 

Meson pole, $p^2 = -m^2$ in the euclidean propagator has \textbf{unit residue} 

The correlation functions of $\phi$ can be expanded in a series of feynman diagrams with momentum
space propagators $\tilde{\Delta}(p;m) = (m^2 + p^2)^-1$. The feynman rules for finite L correlation functions
are the same as in infinite volume except that the space-like components of loop and external momenta
are restricted to discrete values $\textbf{p} = \frac{2\pi}{L}\textbf{n}, n\in \mathbb{Z^3}$

\subsection*{$\mathbb{Z}^3$ Gauge fields on a graph}

It is possible to define gauge fields on an abstract graph $\mathcal{G}$, as on regular lattices. 

In order to study the L dependence of feynman diagrams as $L\rightarrow \infty$, one works in position
space rather than momentum space. The infinite volume propagator is 
$\Delta(x;m) = \int e^{ipx}(m^2 + p^2)^{-1}\frac{d^4p}{(2\pi)^4}$

bound states give rise to poles in the analytically continued forward elastic meson scattering amplitude 

\section*{Meson XPT}

\subsection*{Pion Mass at one-loop}
Using the power counting scheme $\epsilon_\pi = \frac{m_\pi}{\Lambda_\chi} , \Lambda_\chi = 4\pi F $, we compute 
the correction to the pion mass at one loop order. Note that our convention is $F=f\sqrt{2}$.
$\phi= \sum_{i=1}^{3}\phi_i\tau_i$ = $\begin{bmatrix}\frac{\pi^0}{\sqrt{2}} & \pi^+ \\ \pi^- & -\frac{\pi^0}{\sqrt{2}}\end{bmatrix},
 \chi = 2B\hat{m}1$

Starting from the SU(3) matrix $U(x) = exp(i\frac{\phi(x)}{F_0})$, the most general effective Lagrangian Is
$L_{eff} = \frac{F_0^2}{4} Tr(\partial_\mu U \partial^\mu U^\dagger)$. The lagrangian $L_2$ in the 
$SU(2)_L \times SU(2)_R$ is 
$ L_2 = \frac{F^2}{8}Tr(\partial_\mu U \partial^\mu U^\dagger) + \frac{F^2}{8}2B\hat{M}Tr(U + U^\dagger)$. Again,
we are working in the chiral limit where $m_u = m_d = 0$ with $m_s = phys$. When working in SU(3) sector, $m_s = 0$. It is 
also worth noting that in order to obtain the mass of the Goldstone boson, $M_\pi^2 = 2B_0\hat{m}$, we have to work in the 
isospin-symmetric limit $m_u = m_d = \hat{m}$.  
To expand this lagrangian in powers of $\phi$, we begin by substituting in the value of $U$, the SU(3) matrix. 
We obtain 



\section*{Heavy Baryon XPT}
\textit{Baryon Chiral Perturbation Theory} by Jenkins and Manohar

Requirements for usage: Pion momentum is small and baryons are nearly on-shell, the mass of the heavy baryon is irrelevant
in this EFT. 

The octet of the $\frac{1}{2}$ baryons, represented as a traceless $3 \times 3$ matrix $B$ with elements as
complex, four-component Dirac fields, transforms as an octet under the adjoint representation of 
$SU(3)_V$. In order to construct the effective Lagrangian, we start with the group $G = SU(2)_L \times SU(2)_R$.

We want to show that the leading order correction to the nucleon masses are:
\begin{align*}
 &\delta M_p =  +2\alpha_N \frac{2B\delta}{4\pi F_\pi} - 4\sigma_N \frac{M^2_\pi}{4\pi F_\pi} + NNLO+ \\
 &\delta M_n = -2\alpha_N \frac{2B\delta}{4\pi F_\pi} - 4\sigma_N \frac{M^2_\pi}{4\pi F_\pi} + NNLO+
\end{align*}

We begin by constructing the $\pi N$ lagrangian adhering to the transformation rules
of the nucleon and quark fields. 

The covariant derivative 




\subsection*{LO self-energy correction of nucleon masses}

Here, we will compute the leading order correction to the nucleon mass using heavy baryon chiral perturbation theory. First, some notation. We define the spurion field $\chi_{\pm} = u^{\dagger}\chi u^{\dagger} \pm u\chi^{\dagger}u$ where $u^2 = U$.
The nucleon operator is constructed as 
\begin{align*}
  \bar{N}N = \bar{N}^iN_i = [\bar{p}p + \bar{n}N]
\end{align*}
We begin with computing the mass shift from the operator proportional to $\bar{N}Ntr(\chi_+)$. 

\begin{align*}
tr(\chi_+) &= 2tr(\chi) + \dots \\
&= 2tr(2Bm_Q) \\
&= 4(2B\hat{m}) + \dots
\end{align*}

\textbf{partially quenched vs non-quenched vs quenched}

\subsection*{step 2}
From Eq (48) of 0904.2404 as is, the nucleon operators (b1, b5, b6, b8), and compute their correction to the nucleon self-energies.
There are 9 relevant operators in the partially quenched theory, which reduce to 4 in unquenched theory.
We can leave the $T$ terms alone for now. These b terms will condense down to a single parameter when fitting 
lattice data.  Using our new notation, $\mc{M_\pm} \equiv \chi_\pm$; $Tr(\mc{M_\pm})$ evaluates
to a scalar. It is useful to note that $\chi^\delta(m_u = m_d) = 0$. Again, recall that in order to include the 
baryon fields in the chiral lagrangian, we need the relation $u^2=U=exp(\frac{i\pi \centerdot \tau}{F})$. 

As with the mesons, the unquenched baryon chiral lagrangian is given by (with new conventions) 
\begin{widetext}
  \begin{align} \label{eq:LOBaryons}
  \mc{L} =&\ 
    \ol{N} iv \cdot D N 
    +\frac{\alpha_M}{(4\pi F)}\,\ol{N} \chi^\delta_+ N
    +\frac{\sigma_M}{(4\pi F)}\,\ol{N} N \,{\rm tr}(\chi_+) 
  \nonumber\\&\ 
    - (\ol{T}^\mu[iv \cdot D\ - \Delta]T_\mu) 
    +\frac{\alpha_T}{(4\pi F)}\,(\overline{T}^\mu \chi_+^\delta T_\mu)
    +\frac{{\sigma}_T}{(4\pi F)}\,(\ol{T}^\mu T_\mu)\,{\rm tr}(\chi_+)  
  \nonumber\\&\ 
    +\,g_A \,\overline{N}^i S \cdot{u^j_i} \, N_j 
    +\,g_{\Delta\Delta}\,\ol{T}^{kji}_\mu S\cdot {u^{i'}_i}\,T_\mu^{i'jk}  
    +g_{\Delta N}\,\left[\ol{T}^{kji}_\mu \cdot{u}_i^{i'}
                      \epsilon_{j i'} N_k + \overline{N}^k \epsilon^{ij}u^{i'}_i  \cdot T_{i'jk}] \,
  \end{align}
  \end{widetext}

\begin{align}
  \mc{L}_M = \frac{1}{(4\pi f)^3} \bigg\{&
    b_1^M \bar{N}  \, \mc{M}_+^2  N
      + b_5^M \bar{N} N \, \tr ( \mc{M}_+^2 )
      + b_6^M \bar{N} \, \mc{M}_+ N \, \tr (\mc{M}_+) 
      + b_8^M \bar{N} N \, [\tr (\mc{M}_+)]^2
  \nonumber\\&
    + t_1^M \, \bar{T} {}^{kji}_\mu (\mc{M}_+ \mc{M}_+)_{i}{}^{i'} T_{\mu, i'jk} 
    + t_2^M  \, \bar{T} {}^{kji}_\mu (\mc{M}_+)_{i}{}^{i'} (\mc{M}_+)_{j}{}^{j'} T_{\mu, i'j'k} 
    + t_3^M  \, \bar{T}_\mu T_\mu \tr (\mc{M}_+^2)
  \nonumber\\&			
    + t_4^M \, \left( \bar{T}_\mu \mc{M}_+ T_\mu \right) \tr (\mc{M}_+)
    + t_5^M  \, \bar{T}_\mu T_\mu  [ \tr(\mc{M}_+) ]^2
  \nonumber\\&
    +b_1^{W_-} \bar{N} N \Tr (W_- W_-)
  %\nonumber\\&
    +t_1^{W_-} (\bar{T}_\mu T_\mu) \Tr (W_- W_-)
  %\nonumber\\&
    +t_2^{W_-} \bar{T}^{kji}_\mu (W_-)_{i}^{\ i^\prime} (W_-)_{j}^{\ j^\prime} T_{\mu, i^\prime j^\prime k}
    \bigg\}\, .
  \end{align}

Following the process as done by Andre in 2.2 of notes:

Here, we will compute the leading order correction to the nucleon mass using heavy baryon chiral perturbation theory. First, some notation. 
We define the spurion field 
$\chi_{\pm} = u^{\dagger}\chi u^{\dagger} \pm u\chi^{\dagger}u$ where $u^2 = U$.

The nucleon operator is constructed as $\bar{N}N = \bar{N}^iN_i = [\bar{p}p + \bar{n}N]$. 

The set of mass shifts from the four operators in eqn. 48 of 0904 are 
\begin{align*}
\bar{N}Ntr(\chi_+)\chi_+ \\
\bar{N}N(\chi^\delta_+)^2 \\
\bar{N}Ntr(\chi_+)^2  \\
\bar{N}Ntr(\chi^2_+) 
\end{align*}


We begin with computing the mass shift from the operator proportional to $\bar{N}Ntr(\chi_+)\chi_+$ with coefficient $b^M_6$. 

\begin{align*}
tr(\chi_+)\chi_+ &= 2tr(\chi)2\chi + \dots \\
&= 2tr(2Bm_Q)^2 \\
&= 8(4B^2\hat{m}^2) + \dots
\end{align*}

To LO, the full operator in the lagrangian is 
$\mathcal{L} = \frac{\sigma_N}{4\pi F}8(4B^2\hat{m}^2) [\bar{p}p + \bar{n}n] $


The second operator of interest at LO is the $(\chi^\delta_+)^2$ with coefficient $b^1_M$.
\begin{align*}
(\chi^\delta_+)^2 &= (\chi_+ - \frac{1}{2}tr(\chi_+))^2 \\
&= 4\chi^2 - \frac{\chi_+}{4}tr(2\chi) - \frac{\chi_+}{4}tr(2\chi) + \frac{1}{4}tr(4\chi^2) \\
&= 4\chi^2 - \frac{\chi_+}{2}tr(2\chi) + \frac{1}{4}tr(4\chi^2) \\
\end{align*}

The third operator of interest at LO is the $\bar{N}Ntr(\chi^2_+)$ with coefficient $b^5_M$.
\begin{align*}
(\chi^\delta_+)^2 &= (\chi_+ - \frac{1}{2}tr(\chi_+))^2 \\
&= 
  \\
&= 8(4B^2\hat{m}^2) + \dots
\end{align*}

The fourth operator of interst at LO is $\bar{N}Ntr(\chi_+)^2$ with coefficient $b^8_M$
\begin{align*}
tr(\chi_+)^2 &= 4tr(\chi)2\chi + \dots \\
&= 4tr(2Bm_Q)^2 \\
&= 16(4B^2\hat{m}^2) + \dots
\end{align*}
To LO, the full operator in the Lagrangian is 
$\mathcal{L} = \frac{\sigma_N}{4\pi F}16(4B^2\hat{m}^2) [\bar{p}p + \bar{n}n] $

From eqn. 50, $\delta_N = \frac{b^M_5(m^4_\pi)}{2(4\pi F_\pi)^3}$

\subsection*{step 3}
Convert Eq (48) of above, to our updated normalization (from new notes), and repeat

\subsection*{step 4}
Take LO Lagrangian from notes (1.10) and expand up to two pion fields, 
and use the operators found in step 3 to compute the N2LO corrections to the nucleon mass.

Following Andre's notes, the leading order nucleon lagrangian is given by: 
\begin{align*}
  \mc{L} =&\ 
    \ol{N} iv \cdot D N 
    +\frac{\alpha_M}{(4\pi F)}\,\ol{N} \chi^\delta_+ N
    +\frac{\sigma_M}{(4\pi F)}\,\ol{N} N \,{\rm tr}(\chi_+) 
  \nonumber\\&\ 
    - \ol{T}^\mu[iv \cdot D\ - \Delta]T_\mu 
    +\frac{\alpha_T}{(4\pi F)}\,\overline{T}^\mu \chi_+^\delta T_\mu
    +\frac{{\sigma}_T}{(4\pi F)}\,\ol{T}^\mu T_\mu\,{\rm tr}(\chi_+)  
  \nonumber\\&\ 
\end{align*}

The most general Lagrangian under global chiral transformations, 
\begin{align*}
  U \rightarrow LUR^\dagger, X \rightarrow L\chi R^\dagger
\end{align*}

must be the product of terms of the form $Tr(\chi_+\chi^\dagger)$, which remain independent of the pion fields. 
The only invariant term with two derivatives is 
\begin{align*}
  \mathcal{L}_2 = \frac{F^2}{4}tr \partial_\mu \chi \partial^\mu \chi^\dagger.
\end{align*}

This Lagrangian only has terms with an even number of pions because the pion is a pseudoscalar. We can expand $\chi$ in a power series in the pion field as:
\begin{align*}
  \mathcal{L}_2 = Tr \partial_\mu \pi \partial^\mu \pi + \frac{1}{6F^2}Tr[\pi, \partial_\mu \pi]^2 + \dotsm
\end{align*}


At this point, we have to determine the terms in the $\mathcal{L_{eff}}$ that are required to compute to a given order
in $\epsilon_\pi = \frac{m_\pi}{\Lambda_\chi}$. 











\subsection*{meeting 8/20/21}
expansion of nucleon mass splittings should behave as well as for $m_\pi$ itself



\end{document}